
\section{Motivation}

\noindent Interest rates have a influence on various aspects of our lives. They dictate how much we can afford to borrow, how much we earn by saving, but they also influence inflation and unemployment rates \cite{investopedia_interest_rate_new}. Predicting these rates is therefore of great interest to many people and institutions because monetary policy influences the economic developments through expectations \cite{norges_bank_projections}. This implies that we can use the expected path of interest rates in the future to make good decisions today. Unlike a normal stock, an asset that can be traded, interest rates are a feature of an asset. When lending something, e.g. cash, a borrower is charged an interest rate for the use of this asset \cite{investopedia_interest_rate_new}.

When modeling interest rates, we start with the prices of zero-coupon bonds, which is a fixed-income instrument \cite[p.~510--511]{WFI}. Fixed-income instruments, or securities, gives fixed, periodic interest rate payments and returns the principal amount at maturity to an investor \cite{investopedia_fixed_income}. Pricing a bond is harder than pricing an option, which is dependent on an asset, e.g. stocks. This is due to the fact that options have an underlying asset that we can use for hedging. \cite[p.~511]{WFI}. There has been a long journey from the first models trying to fit different characteristics of interest rates to the more sophisticated models we have today. During this time, we have moved from models using one factor to models that can use two or more factors. One of the first models is called the Vasicek Interest Rate Model, and is a one-factor interest rate model \cite{investopedia_vasicek}. A more advanced model is called the HJM Forward Rate Model, and it is a model that can use one or more factors \cite[p.~507--624]{WFI}.

\newpage

\section{Thesis Structure}

\noindent The HJM forward rate model is used to model the whole term structure of interest rates to find the fair value of interest rate derivatives \cite{investopedia_hjm_model}. The term structure of interest rates is also known as the yield curve \cite{investopedia_term_structure}. The HJM forward rate model can therefore capture how each interest rates move together. This makes it an excellent model for evaluating interest rate derivative prices. The analysis of the interest data is shown in Chapter \ref{ch:data}. There are different methods for fitting this model to data, but I will look into a method using PCA to examine the covariance structure of the differenced interest rates, and from this simulate many realizations into the future using the NS model to find the entire yield curve. The NS model is a model for extrapolating or interpolating missing tenors in a yield curve \cite{science_direct_nelson_siegel}. I explain how this is done in Chapter \ref{ch:method}. I therefore aim to find a method that simulates the entire yield curve which is also time efficient, and I will use these simulations to see how the distribution of interest rate derivative prices changes over time. These prices will then be used to look at the risk associated with the derivatives. The simulations, prices, and risk measures are shown in Chapter \ref{ch:results}. The discussion and conclusion are shown in Chapter \ref{ch:disc conc}.

\section{Sustainability}

\noindent Among the United Nations' seventeen sustainability goals, there are two goals that are relevant to this thesis. These are goals 8 and 9. Goal 8 is to "Promote sustained, inclusive and sustainable economic growth, full and productive employment and decent work for all". Because interest rates are used as tools to influence the economic growth and unemployment rates in a country, it is good for people, companies and other countries to have good forecasts of these rates so they can make investments. This is also true for goal 9, which is to "Build resilient infrastructure, promote inclusive and sustainable industrialization and foster innovation". \cite{fn}



\section{HJM Forward Rate Model}

\subsection{Procedures}

\noindent My goal is to find a method that can predict the entire yield curve efficiently and accurately. The reason I want to predict the entire yield curve and not only the tenors I am given is because when making an IRD contract, one first determine what interest rate indices are needed. This will give me greater flexibility when making contracts and evaluating prices. Due to processing power constraints, I define the "entire" yield curve as spanning from $1$ day to $10$ years with the remaining tenors set at three-month intervals. I have come up with two different procedures I want to test. \begin{enumerate}
    \item Using procedure $1$, I first train the HJM model using my data, and then I simulate the interest rates using the Monte Carlo method. These simulations will then be extrapolated and interpolated using the NS model to find the "full" term structure of interest rates. The NS model is from the \textit{YieldCurve} package in \textit{R}.
    \item Using Procedure $2$, I first train the HJM model using my data, then I extrapolate and interpolate the volatilities I obtain. The last date used will be extrapolated and interpolated using the NS model, and by using the volatilities I can then simulate the "full" term structure of interest rates using the Monte Carlo method.
\end{enumerate} I expect that the simulations generated by procedure $1$ will be more accurate than procedure $2$ because I am only using the data I am given. This procedure will take much longer however because for every simulation I need to use the NS model, and this will take a long time depending on how many simulations I use. I expect procedure 2 to be more efficient due to the fact I only use the NS model once in the entire process.

\newpage


\subsection{Data}

\noindent The data will be stored as decimal numbers, i.e. $5\% = 0.05$, in a matrix where the column names are the tenors and each row is a distinct date. This format can be seen in Table \ref{table:rates matrix format} where $t_i$ is date $i$ and $\uptau_j$ is maturity $j$.


\begin{table}[!htbp]
\centering
\begin{tabular}{|c|c|c|c|c|}
\cline{2-5}
\multicolumn{1}{c|}{} & $\uptau_1$ & $\uptau_2$ & $\cdots$ & $\uptau_M$ \\ \hline
$t_1$ & $f \bigl( t_1 , \uptau_1 \bigr)$ & $f \bigl( t_1 , \uptau_2 \bigr)$ & $\cdots$ & $f \bigl( t_1 , \uptau_M \bigr)$ \\ \hline
$t_2$ & $f \bigl( t_2 , \uptau_1 \bigr)$ & $f \bigl( t_2 , \uptau_2 \bigr)$ & $\cdots$ & $f \bigl( t_2 , \uptau_M \bigr)$ \\ \hline
$\vdots$ & $\vdots$ & $\vdots$ & $\ddots$ & $\vdots$ \\ \hline
$t_N$ & $f \bigl( t_N , \uptau_1 \bigr)$ & $f \bigl( t_N , \uptau_2 \bigr)$ & $\cdots$ & $f \bigl( t_N , \uptau_M \bigr)$ \\ \hline
\end{tabular}
\caption[Zero-Coupon Yields Data Format.]{Zero-coupon yields data format.}
\label{table:rates matrix format}
\end{table}


\subsection{Model Assumptions}

\noindent I will assume that the volatilities are only dependent on time to maturity and not dependent on the specific maturity date. Therefore I will use the Musiela parameterization as explained in Section \ref{sec:musiela}. I assume that the volatilities are stationary, i.e. $\overline{\sigma} \bigl( t, \uptau \bigr) = \overline{\sigma} \bigl( \uptau \bigr)$. Because my data is discrete, I will also use the discrete approximation of the HJM model as explained in Section \ref{sec:discrete approx}. This gives the following model: \begin{equation*}
    d \hat{\overline{f}} \bigl( t_i , \uptau_j \bigr) = \hat{\overline{\mu}} \bigl( t_i , \uptau_j \bigr) \bigl[ t_i - t_{i - 1} \bigr] + \sum_{k=1}^D \hat{\overline{\sigma}}_k \bigl( \uptau_j \bigr) \sqrt{t_i - t_{i - 1}} Z_{i,k},
\end{equation*} where \begin{equation*}
    \hat{\overline{\mu}} \bigl( t_i , \uptau_j \bigr) = \sum_{k=1}^D \hat{\overline{\sigma}}_k \bigl( \uptau_j \bigr) \int_{0}^{\uptau_j} \hat{\overline{\sigma}}_k \bigl( s \bigr)ds + \frac{\partial}{\partial \uptau_j} \hat{\overline{f}} \bigl( t_i , \uptau_j \bigr).
\end{equation*} For the shortest tenor, I approximate $\frac{\partial}{\partial \uptau_j} \hat{\overline{f}} \bigl( t_i , \uptau_j \bigr)$ as the slope of the line connecting it to the next tenor. For the longest tenor, I approximate the partial derivative as the slope of the line connected from the prior tenor. For the remaining tenors, I approximate the partial derivative as the average of the slopes connected from the prior tenor and the next tenor. Because $d \hat{\overline{f}} \bigl( t_i , \uptau_j \bigr)$ and $\frac{\partial}{\partial \uptau_j} \hat{\overline{f}} \bigl( t_i , \uptau_j \bigr)$ can be calculated using the data I have they can be collected to the left hand side. This gives the following model: \begin{equation} \label{eq:hjm fit model}
    \begin{split}
        d \hat{\overline{f}} \bigl( t_i , \uptau_j \bigr) - \frac{d}{d \uptau_j} \hat{\overline{f}} &\bigl( t_i , \uptau_j \bigr) \bigl[ t_i - t_{i - 1} \bigr] = \\ \Biggl( \sum_{k=1}^D \hat{\overline{\sigma}}_k \bigl( \uptau_j \bigr) \int_{0}^{\uptau_j} \hat{\overline{\sigma}}_k \bigl( s \bigr)ds \Biggr) \bigl[ t_i - t_{i - 1} \bigr] &+ \sum_{k=1}^D \hat{\overline{\sigma}}_k \bigl( \uptau_j \bigr) \sqrt{t_i - t_{i - 1}} Z_{i,k}
    \end{split}
\end{equation} \newpage \noindent The errors, for each tenor, are therefore assumed to be homoscedastic, and the left hand side of Equation \eqref{eq:hjm fit model} is assumed to be normally distributed with mean equal to the first term of the right hand side of Equation \eqref{eq:hjm fit model} and variance equal to $\sum_{k=1}^D \hat{\overline{\sigma}}_k^2 \bigl( \uptau_j \bigr) \bigl[ t_i - t_{i - 1} \bigr]$.

\subsection{Model Fitting}

\noindent The volatilities in the HJM model are found by first using PCA on the left-hand side of Equation \eqref{eq:hjm fit model}. This is done by first calculating the covariance matrix of this data and multiplying it with $252$. This is because the covariance matrix obtained are for daily changes, so to annualize it I need to multiply it with the number of trading days in a year. From the eigenvalues and eigenvectors of this annualized covariance matrix I can obtain the standard deviations of the principal components and the principal components themselves. The volatilities $\hat{\overline{\sigma}}_k \bigl( \uptau_j \bigr)$ in the HJM model are then the standard deviations of the principal components multiplied with their corresponding principal components. To select the number of factors $D$, or number of principal components, to use I will look at the scree plot obtained.



The volatilities will be fitted using either cubic polynomial regression or natural cubic spline regression with $7$ degrees of freedom, which gives $3$ knots. This gives in total four different models. Two models using procedure $1$ with either polynomial- or spline fitted volatilities, and two models using procedure $2$ with either polynomial- or spline fitted volatilities. The left hand side of Equation \eqref{eq:hjm fit model} is found using integral approximation of the fitted volatilities using the \textit{integrate} function available in \textit{R}.



\subsection{Evaluating Model Fit}

\noindent I will use residual vs fits plots to evaluate if my homoscedastic errors assumption is correct, and I will use Q-Q plots to evaluate my normality assumption. The residuals will be calculated by finding the difference between the left hand side of Equation \eqref{eq:hjm fit model} and the first term of the right hand side.



\subsection{Model Prediction}

\noindent The pseudocode for simulating one realization of the interest rates using the HJM model is shown in Algorithm \ref{code:predict}. To simulate more realization, I just stack these predictions and give them an indicator for which realizations they are. I choose to simulate $10,000$ realizations $10$ years into the future. This is because with $10,000$ realizations I have sufficiently many points to get the true distribution. I choose to simulate $10$ years into the future because it is interesting how much they will change over time.

\newpage

\begin{algorithm}[!htbp]
\SetKwData{Left}{left}\SetKwData{This}{this}\SetKwData{Up}{up}
\SetKwFunction{Union}{Union}\SetKwFunction{FindCompress}{FindCompress}
\SetKwInOut{Input}{input}\SetKwInOut{Output}{output}
\caption{Predicting One Realization using the HJM Forward Rate Model}\label{alg:pred hjm}

\Input{$\boldsymbol{r}_{\text{last}}$ the interest rate data for the last date, \\$n$ number of days to predict, \\$\hat{\overline{\boldsymbol{\mu}}} \bigl( \boldsymbol{\uptau} \bigr)$ fitted drift, \\$\hat{\overline{\mathbf{\sigma}}} \bigl( \boldsymbol{\uptau} \bigr)$ fitted volatilities, \\$d$ number of factors.}
\Output{$\mathbf{r}_\text{pred}$ predicted interest rate data}
$\boldsymbol{r}_{\text{prev}}\gets \boldsymbol{r}_{\text{last}}$\;
$\mathbf{r}_\text{pred} \text{ empty matrix with $n$ rows and the same amount of columns as the training data}$\;
\For{$i = 1$ \KwTo $n$}{
    $\boldsymbol{Z} \sim \mathcal{N}_d \bigl( \boldsymbol{0}, \mathbf{I}_d \bigr)$\;
    
    $\boldsymbol{r}_{\text{next}}\gets \frac{d}{d \boldsymbol{\uptau}} \boldsymbol{r}_{\text{prev}} \cdot dt + \hat{\overline{\boldsymbol{\mu}}} \bigl( \boldsymbol{\uptau} \bigr) \cdot dt + \hat{\overline{\mathbf{\sigma}}} \bigl( \boldsymbol{\uptau} \bigr)^\intercal \cdot \sqrt{dt} \cdot \boldsymbol{Z}$\;
    $\text{Set the $i$'th row of } \mathbf{r}_\text{pred} \text{ to be } \boldsymbol{r}_{\text{next}}$\;
}
\label{code:predict}
\end{algorithm}

\subsection{Evaluating Model Predictions}

\noindent To compare the models I have made, I will reproduce Example 1 in \cite[p.~5--8]{Adin_2024} where they used LGOCV to evaluate two models that predicted values into the future. Group $I_i$ will be all the data at and after time for $y_i$. These groups will be the same for all $i$. So, at the time of $y_{i+1}$, $I_{i+1} = I_i$. The measures I will use to evaluate the prediction error are root-mean-square-prediction-errors, RMSPE, and mean-absolute-prediction-errors, MAPE. The goal of the testing will be how well the model can predict $100$ days into the future. This is because I have a limited number of days to train my model. If I use too much as test data, my model won't pick up the current interest rate characteristics.



\section{Pricing IRDs}

\noindent I want to calculate the value of a fixed-for-floating interest rate swap which has a lifetime of $10$ years, and with swap happening every $6$ months. The value will be calculating by letting the floating part of the swap be a bond with varying payments, and the fixed part will be a bond with fixed payments. They will be discounted daily by using the daily interest rate I obtain from my simulations. The interest rate index I will use is the $6$ month tenor. The fair swap rate will be calculated using Equation \eqref{eq:theory swap rate}. The discounting factors will be calculated using the forward rates from each swap date using Equation \eqref{forward_formula}.
